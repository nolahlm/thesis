%!TEX root = ../main.tex
\chapter{Electronic and Magnetic Characterization of LSMO/LSCO Bilayers}
In order to monitor how interfacial charge transfer and magnetic switching behavior of LSMO / LSCO bilayers is modified by growth on \hkl(1 1 0)$_\mathrm{o}$-oriented NGO substrates, extensive electronic and magnetic characterization was performed.  
This chapter first discusses soft x-ray absorption (XA) and x-ray magnetic circular dichroism (XMCD) results from single layer LSCO, single layer  LSMO, and bilayer LSMO/LSCO on \hkl(1 1 0)$_\mathrm{o}$-oriented NGO substrates including a discussion of fits to experimental XA spectra, and is followed by a discussion of the magnetic switching behavior of these films.  

\section{Experimental Methods}
XA spectra were recorded at beamlines 6.3.1 and 4.0.2 at the Advanced Light Source.  
At beamline 6.3.1, XA spectra were recorded as the average of energy scans with right circularly polarized light in $\pm \SI{1.5}{\tesla}$ fields, while XMCD spectra were taken as the difference.  
Total electron yield (TEY) detection was utilized.  
Co L-edge spectra for samples C12, C8, C4, and C2 as well as Mn L-edge spectra for samples C12M6, C8M6, C4M6, C2M6, and M6 were measured.  
For beamline 4.0.2, XA spectra were taken by averaging energy scans with right or left circularly polarized light in a fixed field of \SI{2}{\tesla}, and both TEY and luminescence yield (LY) detection were used.   
Spectra recorded in TEY mode are representative of the top 4-10\si{\nm} of material because of the finite escape depth of electrons, while those recorded with LY detection provide information from the entire sample volume~\cite{Lee2010,Bianconi1978}.  
Co L-edge spectra for samples C12M6, C8M6, C4M6, and C2M6 were recorded at beamline 4.0.2.  
All x-ray spectra were recorded with x-rays at \SI{30}{\degree} grazing incidence with respect to the sample surface with the applied magnetic field parallel to the incident x-rays, such that the in-plane component of the applied  field was parallel to the NGO \hkl[0 0 1]$_\mathrm{o}$ direction.  

Element specific XMCD hysteresis loops were also measured at beamline 6.3.1.  
Loops at the Co L-edge were measured with $\pm \SI{1.9}{\tesla}$ fields for single layer LSCO films.  
Biased loops at the Mn L-edge were measured for bilayer LSMO/LSCO films where a \SI{1.9}{\tesla} field was applied to the sample to saturate both the soft LSMO and hard LSCO layers.  
Following the application of a saturating field, hysteresis loop were measured up to $\SI{.25}{\tesla}$ field, below the coercivity of the hard layer for bilayer C12M6.  

Major hysteresis loops were measured in a Quantum Design VersaLab physical property measurement system equipped with a vibrating sample magnetometer (VSM) attachment.  
Samples were cooled to \SI{80}{\kelvin} in zero field and the field was swept between $\pm \SI{2}{\tesla}$ while the magnetic moment was recorded.  
The VersaLab was also utilized to measure the magnetic remanence as a function of temperature.  
In these measurements, samples were cooled in a \SI{2}{\tesla} field to \SI{50}{\kelvin}.  
Next the field was set to \SI{0}{\tesla} and the sample was heated to \SI{300}{\kelvin} while the remanent magnetization was monitored.  
All VSM data was normalized to the sample area.  

\section{Soft X-Ray Spectroscopy}
Experiments performed with LSMO/LSCO bilayers on LSAT substrates indicate that a \ce{Co^2+} ion-rich sublayer forms within the LSCO layer close to the LSMO/LSCO interface as a result of charge transfer from LSMO to LSCO~\cite{Li2014}.  
The interfacial sublayer was found to be magnetically soft and strongly coupled with the LSMO layer instead of the hard LSCO layer.  
In order to determine how growth on an orthorhombic substrate with octahedral tilts (\hkl(1 1 0)$_\mathrm{o}$-oriented NGO) impacts charge transfer and formation of the interfacial sublayer, extensive XA spectroscopy  and XMCD measurements were performed at the Co and Mn L-edges.  
\begin{figure}[tb!]
    \centering
    \includegraphics[width=.6\textwidth]{figures/results2/Mn_80k_xas_xmcd_ey.pdf}
    \caption[Mn L-Edge XA and XMCD spectra from LSMO/LSCO bilayers taken at 80K]{Mn L-edge XA and XMCD spectra from LSMO/LSCO bilayers and single layer M6 taken at 80K, acquired in TEY mode. XA spectra shifted vertically for clarity.}
    \label{fig:mn_xas}
\end{figure}
Figure~\ref{fig:mn_xas} plots the experimentally measured Mn L-edge XA and XMCD spectra for all LSMO/LSCO bilayers and includes single layer M6 for comparison.  
XA spectra for all bilayers show close resemblance to sample M6 where the Mn valence state is a mixture of \ce{Mn^4+}/\ce{Mn^3+} ions.  
Charge transfer between individual layers should lead to a change in Mn valence which is accompanied by L$_3$ and L$_2$ peak shifts and changes to spectral shapes; however, no significant differences were observed amongst the Mn spectra.  
The shape of the bilayer LSMO/LSCO XMCD spectra in Figure~\ref{fig:mn_xas} resemble that of single layer M6 indicating that magnetism in the LSMO layer arises from a double exchange interaction between \ce{Mn^3+} and \ce{Mn^4+} ions.  
XMCD spectra for the bilayer samples all have the same magnitude as single layer spectra indicating that the LSMO layer magnetization is constant with changing LSCO layer thickness.  

Experimental Co L-edge XA and XMCD spectra in TEY mode for the LSMO/LSCO bilayers are plotted in Figure~\ref{fig:co_xas_xmcd_ey}.  
Reference spectra from \ce{Co^3+} ions (\ce{LaCoO3}), mixed \ce{Co^3+}/\ce{Co^4+} ions (LSCO), and \ce{Co^2+} ions (\ce{La2CoMnO6}) are also included.  
\begin{figure}[tb!]
    \centering
    \includegraphics[width=.6\textwidth]{figures/results2/Co_bilayer_80k_xmcd_EY.pdf}
    \caption[Co L-Edge XMCD spectra from LSMO/LSCO bilayer taken at \SI{80}{\kelvin} with TEY detection]{Co L-Edge XA and XMCD spectra from LSMO/LSCO bilayers taken at \SI{80}{\kelvin} in TEY mode, with representative reference spectra.  Features A and B denote prominent features associated with \ce{Co^2+} ions and  feature C corresponds with the XA peak from mixed \ce{Co^3+}/\ce{Co^4+}.  Best fits to experimental spectra using Equation~\ref{eq:3_ref} are overlaid in black. XMCD spectra normalized to the absolute value of their minimum}
    \label{fig:co_xas_xmcd_ey}
\end{figure}
Several prominent spectral features are labeled, where features A and B are representative of \ce{Co^2+} ions and feature C corresponding to the mixed \ce{Co^3+}/\ce{Co^4+} L$_3$ peak position.  
Bilayers C12M6 and C8M6 closely resemble the LSCO reference with their L$_3$ peak lining up with feature C, indicating that Co ions are in a mixed \ce{Co^3+}/\ce{Co^4+} valence state in these two films.  
XMCD spectra for C12M6 and C8M6 were also found to be similar in shape to the reference LSCO XMCD spectra, confirming that LSCO layer magnetism arises from double exchange between the mixed valence \ce{Co} ions.  
The Co XA spectra for bilayers C4M6 and C2M6 differ from the two thicker bilayers showing features characteristic of \ce{Co^2+} ions - the L$_2$ peak becomes highly asymmetric while the L$_3$ peak shifts to lower energies and shows a complex multiplet structure.  
The observed spectral changes for the two thinnest bilayers are representative of charge transfer from the LSMO layer to the LSCO layer.  
XMCD spectra for bilayers C4M6 and C2M6 show a small positive peak and large negative peak lining up with the reference \ce{La2CoMnO6} XMCD spectra indicating that the \ce{Co^2+} ions are magnetically active.  

The Co XA and XMCD spectra in Figure~\ref{fig:co_xas_xmcd_ey} were taken in TEY mode and therefore give information primarily from the LSCO layer close to the LSMO interface where charge transfer occurs.  
Spectra were also recorded in LY mode giving information from throughout the volume of the LSCO layer.  
LY spectra are shown in Figure~\ref{fig:co_xas_ly}.  
The LY efficiency of the NGO substrates is very small resulting in a low signal-to-noise ratio especially for the thinnest LSCO samples and XMCD measurements.  
For this reason, XMCD spectra in LY mode are not shown.  
\begin{figure}[tb!]
    \centering
    \includegraphics[width=.6\textwidth]{figures/results2/Co_bilayer_80k_xmcd_LY.pdf}
    \caption[Co L-Edge XA spectra from LSMO/LSCO bilayers taken at \SI{80}{\kelvin} with LY detection]{Co L-Edge XA spectra from LSMO/LSCO bilayers taken at \SI{80}{\kelvin} with LY detection, with representative reference spectra (* denotes TEY detection).  Best fits overlaid in black.  Features A and B denote prominent features associated with \ce{Co^2+} ions and  feature C corresponds with the XA peak from mixed \ce{Co^3+}/\ce{Co^4+}.}
    \label{fig:co_xas_ly}
\end{figure}
LY Co L-edge XA spectra for bilayers C12M6 and C8M6 closely resemble the TEY spectra with their L$_3$ peak lining up with that of the LSCO reference.  
As the Co layer thickness decreases, features A and B begin to grow, although to a lesser extent than in the TEY spectra.  
The complex multiplet structure observed in TEY spectra is also not present and instead bilayers C4M6 and C2M6 are closer in shape to the \ce{Co^3+} \ce{LaCoO3} reference.  
MEPS increases the amount of \ce{Co^3+} ions compared with \ce{Co^4+} ions in close proximity to the substrate interface, a volume of material that is only probed in LY mode for the LSMO/LSCO bilayers.  
XA spectra for single layer LSCO films in Figure~\ref{fig:co_80k_xa_xmcd_single}, taken in TEY mode, also resemble the \ce{LaCoO3} for thickness \SI{4}{\nm} and below, in agreement with measurements performed for similar films on LSAT substrates~\cite{Li2017b}.  
The two thicker LSCO films closely resemble the LSCO reference spectra.  
\begin{figure}[tb!]
    \centering
    \includegraphics[width=.6\textwidth]{figures/results2/Co_single_80k_xmcd_ey.pdf}
    \caption[Co L-edge XA and XMCD spectra from single layer Co samples taken at 80K]{Co L-edge XA and XMCD spectra taken at 80K from single layer Co samples in TEY mode, compared with reference spectra representative of mixed \ce{Co^3+}/\ce{Co^4+} (LSCO) and \ce{Co^3+} (\ce{LaCoO3}) valence states.  Black lines show the best fit to experimental data, taken as the linear combination of reference spectra in Equation~\ref{eq:3_ref}.  Samples C4 and C2 showed extensive charging effects and their XMCD spectra are not shown.}
    \label{fig:co_80k_xa_xmcd_single}
\end{figure}
No \ce{Co^2+} ion formation is observed for the single layer samples confirming that its formation depends on the presence of an LSMO overlayer.  
XMCD spectra for single layers C12 and C8 closely resemble the LSCO reference, while single layers C4 and C2 showed significant charging effects and their XMCD spectra are therefore not shown.  

\section{X-Ray Absorption Spectra Fitting}
To extract quantitative information about ionic valence changes and better compare XA results between different samples and detection modes, experimental spectra were fit using a linear combination of reference spectra.  
To perform the fits, experimental spectra must first have their background removed.  
The background is given by a step function convoluted with a Gaussian, which is estimated by an error function~\cite{vanderLaan2014}.  
For a pure metal sample, background subtraction is relatively straightforward: the background shows two jumps as it passes through the L$_3$ and L$_2$ peaks, with the L$_3$ background height being double that of the L$_2$ height.  
The total height of the jump can be taken as the signal intensity after any extended absorption oscillations have subsided, which occurs approximately 20-30\si{\eV} beyond the L-edge~\cite{Chen1995a}.  
The presence of \ce{La} ions in LSCO complicates the analysis because the \ce{La} M-edge is only \SI{40}{\eV} away from the \ce{Co} L-edge and the absorption effects at either edge cannot be isolated.  
For the purposes of this work, the first jump extends from the pre-edge to the XA intensity between the L$_3$ and L$_2$ peaks, and the second background jump goes up to the height of the post edge, as depicted in Figure~\ref{fig:xa_backsub_example} for the Co XA spectrum from bilayer C12M6.  
\begin{figure}[tb!]
    \centering
    \includegraphics[width=.6\textwidth]{figures/results2/backsub_example.pdf}
    \caption[Example of XAS Background Subtraction]{Background subtraction for bilayer C12M6.  Two stitched error functions are used to estimate the background.}
    \label{fig:xa_backsub_example}
\end{figure}
It is also crucial to ensure that there are no energy shifts present between experimental and reference spectra which is done by lining up prominent spectral features amongst experimental and reference spectra.  

Two different linear combinations of reference spectra were used to fit the XA spectra:
\begin{align}
    I &= A \times I_\mathrm{LSCO} + (1-A) \times I_{\ce{La2CoMnO6}} \label{eq:2_ref}\\
    I &= A \times I_\mathrm{LSCO} + B \times I_{\ce{La2CoMnO6}} + C \times I_{\ce{LaCoO3}} \label{eq:3_ref}
\end{align}
where Equation~\ref{eq:3_ref} adds sensitivity to pure \ce{Co^3+} valence compared with Equation~\ref{eq:2_ref}.  
Fit results using Equation~\ref{eq:3_ref} are overlaid as black curves in Figures~\ref{fig:co_xas_xmcd_ey},~\ref{fig:co_xas_ly}, and~\ref{fig:co_80k_xa_xmcd_single}.  
Using all three reference spectra results in better fits for all samples measured based on a $\chi^2$ test as shown in Table~\ref{tab:goodness_of_fit}.  
\begin{table}[tb]
\centering
\caption{Goodness of fit values for XA spectra fitting of single layer LSCO and bilayer LSMO/LSCO samples, simulated with either two reference spectra (equation~\ref{eq:2_ref}) or three reference spectra (equation~\ref{eq:3_ref}).}
\label{tab:goodness_of_fit}
\begin{tabular}{@{}llllllllll@{}}
\toprule
&& \multicolumn{2}{c}{Single Layer} && \multicolumn{2}{c}{Bilayer TEY} && \multicolumn{2}{c}{Bilayer LY}\\
\cmidrule{3-4} \cmidrule{6-7} \cmidrule{9-10}
LSCO Thickness && $\chi^2$ (\ref{eq:2_ref}) & $\chi^2$ (\ref{eq:3_ref}) && $\chi^2$ (\ref{eq:2_ref}) & $\chi^2$ (\ref{eq:3_ref}) && $\chi^2$ (\ref{eq:2_ref}) & $\chi^2$ (\ref{eq:3_ref}) \\ 
\hline
\SI{12}{\nm} && 0.204 & 0.174 && 0.238 &  0.221 && 0.483 & 0.470 \\
\SI{8}{\nm} && 0.278 & 0.198 && 0.150 & 0.148 && 0.314 & 0.292 \\
\SI{4}{\nm} && 0.649 & 0.192 && 0.233 & 0.148 && 0.274 & 0.208 \\
\SI{2}{\nm} && 2.72 & 0.472 && 0.464 & 0.193 && 1.99 & 0.940 \\
\bottomrule
\end{tabular}
\end{table}
Fits for samples with LSCO thicknesses of \SI{2}{\nm} and \SI{4}{\nm} show the largest change when incorporating all three reference spectra, in agreement with the qualitative observation that the \ce{Co^3+}/\ce{Co^4+} ratio increases as LSCO thickness decreases below \SI{8}{\nm}~\cite{Li2017b}.  

Figure~\ref{fig:fit_weights} presents the weight of the different reference spectra using Equation~\ref{eq:3_ref} for the bilayer spectra in TEY and LY mode, as well as singe layer spectra in TEY mode.  
\begin{figure}[tb!]
\centering
    \begin{subfigure}[b]{0.45\textwidth}
         \centering
         \includegraphics[width=\textwidth]{figures/results2/lsco_lcmo_lco_weights.pdf}
         \caption{Bilayer LSMO/LSCO, TEY.}
         \label{fig:weights_bilayerTEY}
    \end{subfigure}
    \hfill
    \begin{subfigure}[b]{0.45\textwidth}
         \centering
         \includegraphics[width=\textwidth]{figures/results2/LY_lsco_lcmo_lco_weights.pdf}
         \caption{Bilayer LSMO/LSCO, LY.}
         \label{fig:weights_bilayerLY}
    \end{subfigure}
    \\
    \begin{subfigure}[b]{0.45\textwidth}
         \centering
         \includegraphics[width=\textwidth]{figures/results2/lsco_lcmo_lco_singlelayer_weights.pdf}
         \caption{Single layer LSCO, TEY.}
         \label{fig:weights_co_single}
    \end{subfigure}
\caption[Spectral weights for XAS fits]{Spectral weights for XAS fits using Equation~\ref{eq:3_ref}.}
\label{fig:fit_weights}
\end{figure}
Results from the fits for single layer LSCO films show a transfer of weight from the \ce{Co^3+} ion reference to the LSCO reference as the LSCO layer thickness increases, with little \ce{Co^2+} ion signal.  
The large amount of \ce{Co^3+} ion weight for single layer C2 confirms substrate-induced MEPS, while the reduction in \ce{Co^3+} ion intensity with increasing thickness is an effect of the finite sample volume of TEY mode.  
As the LSCO thickness increases a \ce{Co^3+}/\ce{Co^4+} layer forms reducing the volume of MEPS layer probed by TEY detection.  

The results from fitting the TEY spectra for bilayers in Figure~\ref{fig:weights_bilayerTEY} agree with qualitative observations suggesting formation of \ce{Co^2+} ions when the LSCO thickness decreases below \SI{8}{\nm}.  
Weights of the \ce{Co^2+} reference decrease from $0.78$ for C2M6 to $0.55$ for C4M6, and reaches zero for C8M6.  
The LSCO reference weight follows an opposite trend, increasing from 0 for C2M6 to \SI{.97}{} for C8M6.  
Fits of LY spectra in Figure~\ref{fig:weights_bilayerLY} are similar to results from the TEY spectra, being predominantly LSCO-like with no \ce{Co^2+} formation and small amounts of \ce{Co^3+}.  
Differences in reference spectra weights between TEY and LY mode are apparent for the two thinner bilayers.  
The \ce{Co^3+} weight for bilayer C2M6 is \SI{50}{\percent} larger for the fits using LY spectra while the \ce{Co^2+} weight is \SI{60}{\percent} lower.  
Bilayer C4M6 also shows a large reduction in \ce{Co^2+} weight, but this difference is balanced by an increase in the LSCO weighting with no change in \ce{Co^3+} for the LY spectra fits indicating formation of a \ce{Co^3}/\ce{Co^4+} layer.  
These differences between LY and TEY spectra arise because of the different sample volumes probed, with LY spectra being representative of a the entire LSCO sublayer volume gaining sensitivity to the LSCO/substrate interface.  

Summarizing the fitting results, single layer LSCO splits into multiple layers with a non-magnetic MEPS layer at the LSCO-substrate interface.  
For single layer C2 there is only a MEPS layer present, while for LSCO layer thicknesses of \SI{4}{\nm} and  above an additional mixed valence \ce{Co^3+}/\ce{Co^4+} LSCO layer is also present with magnetism mediated by an double exchange.  
The LSCO layer in LSMO / LSCO bilayers displays different behavior as a function of thickness.  
Bilayer C2M6 has two LSCO layers with MEPS at the LSCO-substrate interface and a \ce{Co^2+} ion-rich magnetically active interfacial layer.  
For bilayer C4M6 an additional LSCO layer forms between the MEPS and \ce{Co^2+} ion-rich layer.  
Interestingly, the \ce{Co^2+} interfacial layer is not present for bilayers C8M6 and C12M6, even based on TEY detection data which is most sensitive to the LSMO / LSCO interfacial region.  
These results for bilayers are in contrast with similar experiments performed on LSAT substrates that show a more gradual reduction in the \ce{Co^2+} ion-rich layer as a function of increasing LSCO layer thickness, persisting for LSCO layer thicknesses of \SI{8}{\nm}~\cite{Li2016,Kane2019}.  

\section{Magnetometry}
Major hysteresis loops were measured for all bilayers as well as single layers C12, C8, and M6 at \SI{80}{\kelvin} to a  $\pm \SI{2}{\tesla}$ measuring field along the NGO \hkl[0 0 1]$_\mathrm{o}$ direction to ensure saturation of both magnetic layers.  
Single layers C4 and C2 were found to be non-magnetic and no hysteresis loop were observed.  
Hysteresis loops were normalized to the sample area unless otherwise noted.  
The remanent magnetization, $M_r$, was also tracked as a function of temperature.  
Typical magnetization versus temperature measurements that apply a small measuring field are difficult to perform for films on NGO substrates because of their large paramagnetic moment and therefore the remanent magnetization was tracked as a function of temperature.  
The $M_r(T)$ data was collected by cooling the sample in a field \SI{2}{\tesla} parallel to the NGO \hkl[0 0 1]$_{\mathrm{o}}$ direction and measuring the magnetic moment in zero applied field while the sample is heated.  

The T-dependence of $M_r$ is shown in Figure~\ref{fig:bilayer_tmr} for all four bilayers as well as single layers M6, C12, and C8.  
\begin{figure}[tb!]
    \centering
    \includegraphics[width=.7\textwidth]{figures/results2/bilayer_tmr.pdf}
    \caption[Remanent magnetization versus temperature]{Remanent magnetization versus temperature for LSMO/LSCO bilayer series and single layers M6, C12, and C8.  Cooling field and magnetization measured along the NGO \hkl[0 0 1]$_\mathrm{o}$ direction.}
    \label{fig:bilayer_tmr}
\end{figure}
Similar profiles were observed for all measurements except that of bilayer C2M6 which shows little magnetization for all measured temperatures.  
A sharp drop to negative remanence was seen as the temperature was raised to a critical value.  
This drop to negative magnetization may be attributed to small amounts of trapped flux in the magnet yoke.  
When temperature is increased to approach $T_c$, the coercive field typically decreases~\cite{Lee2017} and eventually any small amount of trapped flux is able to reverse the sample magnetization leading to a negative $M_r$.  

A critical temperature can be defined for the $M_r$ data as the point where $dM_r/dT$ is at a maximum.  
Values are reported in Table~\ref{tab:bialyer_tc_tmr} for all samples.  
\begin{table}[tb!]
    \centering
    \caption{Critical Temperature extracted from $M_r$ temperature dependence.}
    \begin{tabular}{ll}
    \toprule
    Sample & T$_\mathrm{c}$ (\si{\kelvin})    \\ \hline
    C12M6  & 194.7 \\ 
    C12 & 190.7 \\
    C8M6   & 180.8 \\ 
    C8 & 174.9 \\
    C4M6   & 239.5 \\ 
    C2M6 & 123.6 \\
    M6     & 270.0 \\ 
    \bottomrule
    \end{tabular}
    \label{tab:bialyer_tc_tmr}
\end{table}
For bilayer C12M6, the critical point occurs \SI{4}{\kelvin} above single layer C12 and for bilayer C8M6 it occurs \SI{5.9}{\kelvin} above that of single layer C8.  
Interestingly the remanent magnetization for bilayers drops to zero at the critical temperature.  
A small amount of remanent magnetization associated with the LSMO layer may be expected to persist above the critical temperature for the bilayer samples as hysteresis loops recorded at \SI{80}{\kelvin} and \SI{250}{\kelvin} for bilayer C12M6, shown in Figure~\ref{fig:c12m6_Temploops}, indicate that the two thicker bilayer samples are still magnetic above their single reported critical temperatures with the second hard magnetic transition at \SI{0.6}{\tesla} only occurring below \SI{80}{\kelvin}.  
These hysteresis loop results suggest that the critical temperatures recorded for bilayers C12M6 and C8M6 are the temperatures at which the LSCO layer becomes nonmagnetic.  

Bilayer C4M6 differs from the two thicker bilayer samples with a critical temperature within \SI{40}{\kelvin} of the critical temperature for M6.  
XA and XMCD results indicate that bilayer C4M6 forms a magnetically active \ce{Co^2+} ion-rich interfacial layer which may account for this difference.  
Bilayer C2M6 has a very weak remanent magnetization with a low critical temperature of \SI{123.6}{\kelvin}.  
\begin{figure}[tb!]
\centering
\begin{subfigure}[b]{0.45\textwidth}
         \centering
         \includegraphics[width=\textwidth]{figures/results2/80k_250k_na08_loops.pdf}
         \caption{C12M6}
         \label{fig:c12m6_Temploops}
     \end{subfigure}
     ~
     \begin{subfigure}[b]{0.45\textwidth}
         \centering
         \includegraphics[width=\textwidth]{figures/results2/c4m6_temp_loops.pdf}
         \caption{C4M6}
         \label{fig:c4m6_temploops}
     \end{subfigure}
\caption{(a) Hysteresis loops for bilayer C12M6 at \SI{80}{\kelvin} and \SI{250}{\kelvin}.  Magnetic field applied along the NGO \hkl[1 -1 0]$_{o}$ direction (b) Hysteresis loops for bilayer C4M6 at \SI{80}{\kelvin}, \SI{225}{\kelvin}, and \SI{300}{\kelvin}.  Magnetic field applied along the NGO \hkl[0 0 1]$_{o}$ direction}
\label{fig:bilayer_temp_loops}
\end{figure}
Hysteresis loops for bilayer C4M6 in Figure~\ref{fig:c4m6_temploops} show that it is non-magnetic above the critical temperature of \SI{239.5}{\kelvin} confirming that for this bilayer the critical temperature represents the magnetic transition for both layers.  
Coincident loss of magnetization in both the LSMO and LSCO layers in bilayer C4M6 suggests that the two layers are magnetically coupled, as does the observation that the hysteresis loop for C4M6 at \SI{80}{\kelvin} shows just one magnetic transition.  
An interfacial layer with magnetically active \ce{Co^2+} ions, shown to be present in bilayer C4M6 by XA and XMCD spectroscopy, tends to couple strongly to the LSMO layer~\cite{Li2014}.  

Major hysteresis loops were recorded for all LSMO/LSCO bilayers as well as single layers C12 and C8 in order to track how magnetic switching varies as a function of LSCO thickness.  
In a VSM measurement, the recorded dataset is a combination of all of the magnetic signals present in the chamber including those from the thin films of interest and the substrate.  
NGO substrates have a strong paramagnetic signal because of the \ce{Nd} ions and it is often reported that VSM magnetometry is not possible for thin films grown on this substrate~\cite{Bolstad2018}, but the results in this thesis suggest that this is not necessarily true.  

In Figure~\ref{fig:loop_backsub} the background subtraction process is illustrated using single layer C12 as a representative example.  
\begin{figure}[tb]
\centering
\begin{subfigure}[b]{0.45\textwidth}
         \centering
         \includegraphics[width=\textwidth]{figures/results2/loop_backsub_example.pdf}
         \caption{}
         \label{fig:loop_backsub}
     \end{subfigure}
     ~
     \begin{subfigure}[b]{0.45\textwidth}
         \centering
         \includegraphics[width=\textwidth]{figures/results2/c12_xmcd_loop_001.pdf}
         \caption{}
         \label{fig:c12_xmcd_loop}
     \end{subfigure}
\caption{(a) Background subtraction procedure for single layer C12.  
        (b) C12 Co XMCD hysteresis loop.  Magnetic field applied along NGO \hkl[0 0 1]$_{o}$ direction}
\label{fig:backsub_xmcd_loop}
\end{figure}
First the raw data (red line) is fit to a straight line and subtracted from the dataset, resulting in the blue curve which shows an unexpected soft magnetic transition at low fields.  
A small residual ferromagnetic signal from either the substrate or VSM chamber is responsible for the observed soft switching event, gathered by running an identical scan on a \ce{Pd} reference standard and subtracting away the \ce{Pd} paramagnetic moment.  
Ideally this scan would be performed on an identical piece of NGO.  
When the residual ferromagnetic signal is removed, the green curve in Figure~\ref{fig:loop_backsub} is recovered showing a single hard switching event for single layer C12.  
The Co elemental specific hysteresis loop in Figure~\ref{fig:c12_xmcd_loop} justifies subtraction of a residual loop because it only shows a single hard magnetic transition with a coercivity of \SI{0.70}{\tesla}.  
Data divergence at high magnetic fields for the XMCD hysteresis loop is attributed to interactions between the changing magnetic field and the captured TEY and I0 normalization signals.  
After background subtraction the coercive field obtained from VSM measurements for single layer C12 is \SI{0.75}{\tesla}, in close agreement with the value extracted from XMCD loops.  

Hysteresis loops for singe layers C12 and C8 are plotted in Figure~\ref{fig:loops_single_layer}.  
\begin{figure}[tb!]
    \centering
    \includegraphics[width=\textwidth]{figures/results2/single_loops_final.pdf}
    \caption{Hysteresis loops for single layer LSCO at \SI{80}{\kelvin} measured in a VSM (left panel) and XMCD hysteresis loops at the Co L-edge (right panel).  Magnetic field applied along the NGO \hkl[0 0 1]$_{o}$ direction}
    \label{fig:loops_single_layer}
\end{figure}
The background subtraction procedure successfully removed the soft magnetic transition for single layer C12 but did not work completely for single layer C8, and therefore assuming a constant background amongst different samples is not completely accurate.  
Instead, each substrate piece should be measured for background subtraction prior to film growth in order to account for magnetic differences amongst substrate pieces although doing so risks surface contamination which negatively impacts film growth.  
In this way, an accurate background value can be determined for each sample.   
Single layer C8 shows a soft transition with a coercivity of \SI{0.02}{\tesla} and a hard transition with a coercivity of \SI{.32}{\tesla}.  
This hard magnetic transition is also present in the XMCD loop at the same coercive field.  

Remanence values for single layer C12 agree with the values from the temperature dependence in Figure~\ref{fig:bilayer_tmr}, whereas the remanence value extracted from the loops for C8 is double that of the value from the temperature dependent measurements.  
The observed discrepancy in $M_r$ between the hysteresis loop and the $M_r$ versus T data is another indicator that the background removal attempts were not successful for single layer C8.  

Versalab loops for all four bilayers are presented in Figure~\ref{fig:bilayer_vsm_loops}.  
\begin{figure}[tb!]
    \centering
    \includegraphics[width=\textwidth]{figures/results2/bilayer_loops_final.pdf}
    \caption[Versalab Hysteresis loops for LSMO/LSCO bilayers]{Hysteresis loops for LSMO/LSCO bilayers measured at \SI{80}{\kelvin}.  Magnetic field applied along the NGO \hkl[0 0 1]$_{o}$ direction}
    \label{fig:bilayer_vsm_loops}
\end{figure}
The hysteresis loop for bilayer C12M6 shows two switching events, a soft switching event at \SI{.077}{\tesla} corresponding with the LSMO layer and a hard switching event at \SI{0.6}{\tesla} which is \SI{0.15}{\tesla} lower than the coercivity for single layer C12.  
Bilayer C8M6 also shows two switching events with similar coercive fields while bilayers C4M6 and C2M6 show just one magnetic switching event.  
Bilayer C4M6 has a square hysteresis loop and a large $M_r$ while bilayer C2M6 has negligible remanence consistent with a magnetic hard direction.  
Magnetic remanence trends from hysteresis loops of LSMO/LSCO bilayers are consistent with those from $M_r$ as a function of temperature: as the LSCO sublayer thickness decreases $M_r$ also decreases being close to zero for \SI{2}{\nm} LSCO layer thickness.  
Saturation magnetization for bilayer C8M6 is larger than that for C12M6 while it is nearly the same for bilayers C4M6 and C2M6 even though areal normalized $M_s$ values should scale with total film thickness.  
Mn L-edge XMCD measurements show identical XMCD signal from these two bilayers indicating that the discrepancy in saturation magnetization is due to contributions from the LSCO layer.  
The Co L-edge XMCD maximum in TEY mode for bilayer C4M6 is only \SI{10}{\percent} smaller than that of bilayer C2M6, therefore the interfacial layer cannot be responsible for the observed difference.  

Hysteresis loops measured on LSAT substrates for bilayers C12M6 and C8M6 show two distinct magnetic transitions with the soft magnetic transition occurring at \SI{.0006}{\tesla} for C8M6 and \SI{.0009}{\tesla} for C12M6, significantly lower than that for the same bilayers on NGO substrates.  
The hard switching event for bilayer C12M6 on LSAT and NGO substrates occurs at roughly the same switching field of \SI{.6}{\tesla}.  
There are also distinct differences between the two substrates for the thinner bilayers.  
On LSAT substrates the loop shape for bilayers C4M6 and C2M6 are the same showing a single soft switching event representative of strong magnetic coupling between the LSCO and LSMO layers~\cite{Wynn2015}.  
Hysteresis loops for these two bilayers on NGO substrates also show a single switching event but bilayer C2M6 has little remanence along the \hkl[0 0 1]$_\mathrm{o}$ direction.  
The ratio between the remanent magnetization and the saturation magnetization, referred to as the loop squareness, is given for LSMO/LSCO bilayers on NGO and LSAT substrates in Figure~\ref{tab:bialyer_mr/ms}.    
\begin{table}[tb!]
    \centering
    \caption{$M_r / M_s$ ratio for the LSMO/LSCO bilayer hysteresis loops in Figure~\ref{fig:bilayer_vsm_loops}}
    \begin{tabular}{lll}
    \toprule
    Sample & $M_r / M_s$ (NGO) &  $M_r / M_s$ (LSAT)~\cite{Wynn2015}  \\ \hline
    C12M6  & 0.652 & 0.59\\ 
    C8M6   & 0.469 & 0.84\\ 
    C4M6   & 0.942 & 0.45\\ 
    C2M6   & 0.256 & 0.60\\ 
    \bottomrule
    \end{tabular}
    \label{tab:bialyer_mr/ms}
\end{table}
The loop squareness is a useful comparison for comparing how hard or soft a magnetic transition is, with a small value indicating a magnetically hard direction.  
No obvious trend exists for the loop for bilayers on NGO and LSAT substrates suggesting that multiple different competing factors control magnetic switching in LSMO/LSCO bilayers.  

\section{Magnetocrystalline Anisotropy}
Hysteresis loops were measured for bilayer C12M6 with the field applied parallel to the \hkl[1 0 0]$_{pc}$, [0 1 0]$_{pc}$, and [1 1 0]$_{pc}$ directions, while XMCD hysteresis loops, were recorded along the \hkl[1 0 0]$_{pc}$ and \hkl[1 1 0]$_{pc}$ directions for single layers C12, C8, and M6 to study magnetocrystalline anisotropy in films on NGO substrates.  
The loops for bilayer C12M6, shown in Figure~\ref{fig:c12m6_anisotropy_loops}, have different shapes when measured along the \hkl[1 0 0 ]$_{pc}$ direction compared with the  \hkl[0 1 0]$_{pc}$ and \hkl[1 1 0]$_{pc}$ directions.  
\begin{figure}[tb]
    \centering
    \includegraphics[width=.5\textwidth]{figures/results2/c12m6_anisotropy.pdf}
    \caption{Hysteresis loops for bilayer C12M6 measured at \SI{80}{\kelvin} with the magnetic field applied parallel to the labeled direction}
    \label{fig:c12m6_anisotropy_loops}
\end{figure}
When the field is applied parallel to the \hkl[1 0 0]$_{pc}$ direction, the hard transition at \SI{.58}{\tesla} corresponding with the LSCO layer is significantly sharper compared with the very gradual transition measured along the other two pseudocubic directions.  
These results for bilayer C12M6 suggest a \hkl[1 0 0]$_{pc}$ easy direction.  

Co L-edge XMCD hysteresis loops for single layers C12 and C8 are shown in Figure~\ref{fig:xmcd_singlelayer_loops}.  
All of the loops show nonlinear behavior at high fields which is attributed to the changing magnetic field altering the raw TEY detection signal.  
Furthermore, the I0 incident beam intensity normalization signal shows unexpected magnetic field dependence which further alters the loop behavior at high fields.  
\begin{figure}[tb]
\centering
    \begin{subfigure}[tb]{.49\textwidth}
        \centering
        \includegraphics[width=\textwidth]{figures/results2/c12_xmcd_loop_anisotropy.pdf}
        \caption{}
        \label{fig:c12_xmcd}
    \end{subfigure}
    ~
    \begin{subfigure}[tb]{.49\textwidth}
        \centering
        \includegraphics[width=\textwidth]{figures/results2/c8_xmcd_loop_anisotropy.pdf}
        \caption{}
        \label{fig:c8_xmcd}
    \end{subfigure}
\caption{Co XMCD hysteresis loops for (a) C12 and (b) C8.  Legend denotes direction of applied magnetic field.}
\label{fig:xmcd_singlelayer_loops}
\end{figure}
For both C12 and C8 a single hard switching event was observed for both measured orientations.  
In C12 and C8 there is a \SI{20}{\percent} reduction in the coercive field between the \hkl[1 0 0]$_\mathrm{pc}$ and \hkl[1 1 0]$_\mathrm{pc}$  directions suggesting a \hkl[1 1 0]$_\mathrm{pc}$ easy direction.  
The shape of XMCD spectra does not change for different orientations, while a \SI{30}{\percent} reduction in the XMCD maximum is observed along the \hkl[1 1 0] direction for both single layers C12 and C8.  
Furthermore the coercive field decreases as the LSCO thickness decreases.  
These observations are consistent with the thickness and orientation trends in coercivity for single layer LSCO on LSAT substrates~\cite{Wynn2015}.   

The Mn L-edge XMCD loops for single layer M6 are shown in Figure~\ref{fig:m6_xmcD_loop}.  
Mn XA and XMCD spectra are identical in shape and magnitude along the two measured directions for single layer M6.  
A single magnetic transition is observed along both directions, and the loop shape is nearly identical.  
\begin{figure}[tb]
    \centering
    \includegraphics[width=.5\textwidth]{figures/results2/m6_xmcd_loop_anisotropy.pdf}
    \caption{Mn XMCD Hysteresis loop for single layer M6.  Legend denotes direction of applied magnetic field}
    \label{fig:m6_xmcD_loop}
\end{figure}
Furthermore, no change in coercivity occurs indicating a lack of magnetocrystalline anisotropy.  
This observation is in direct contrast to films on NGO and LSAT substrates.   
\SI{18}{\nm} LSMO films on NGO possess a \hkl[1 0 0]$_\mathrm{pc}$ easy direction and a \hkl[0 1 0] hard direction~\cite{Matthews2007}.  
LSMO on LSAT substrates show uniaxial magnetic anisotropy with a \hkl[1 1 0] easy direction and a \hkl[1 0 0] hard direction, although the anisotropy is not strong~\cite{Monsen2014}.  

Summarizing the magnetocrystalline anisotropy results, bilayer C12M6 has a \hkl[1 0 0]$_{pc}$ easy direction.  
In contrast, single layers C12 and C8 have a \hkl[1 1 0]$_{pc}$ easy direction while single layer M6 does not have anisotropic magnetic behavior.  
These results show that both film thickness and the presence of multiple magnetic layers have a strong impact on magnetocrystalline anisotropy for films grown on NGO substrates.  

\section{Biased Minor Hysteresis Loops}
Biased Mn L-edge XMCD minor loops were measured for bilayer C12M6, shown in Figure~\ref{fig:c12m6_biased_minor}, where a $\pm \SI{1.9}{\tesla}$ biasing field was applied.  
The minor loops were measured up to \SI{0.25}{\tesla}, below the coercive field of the LSCO layer along the \hkl[1 0 0]$_\mathrm{pc}$ direction ensuring that only the soft LSMO layer was switched.  
The same maximum field for the minor loops was used for measurements along the two crystallographic directions measured.  
If the \SI{20}{\percent} reduction of coercivity for single layer C12 along the \hkl[1 1 0]$_\mathrm{pc}$ is assumed to still hold for bilayer C12M6, a \SI{0.20}{\tesla} field remains insufficient to switch the LSCO layer.  
\begin{figure}[tb!]
\centering
    \begin{subfigure}[b]{.49\textwidth}
        \centering
        \includegraphics[width=\textwidth]{figures/results2/c12m6_minor_pos.pdf}
        \caption{}
        \label{fig:c12m6_minor_pos}
    \end{subfigure}
    ~
    \begin{subfigure}[b]{.49\textwidth}
        \centering
        \includegraphics[width=\textwidth]{figures/results2/c12m6_minor_neg.pdf}
        \caption{}
        \label{fig:c12m6_minor_neg}
    \end{subfigure}
\caption{Mn L-edge XMCD loops for bilayer C12M6 after biasing with (a) \SI{1.9}{\tesla} and (b) \SI{-1.9}{\tesla}}
\label{fig:c12m6_biased_minor}
\end{figure}
There is an obvious change in shape between the two directions: when measured along the \hkl[1 0 0]$_\mathrm{pc}$ direction the minor loops are significantly tilted which indicates pinning of magnetic moments in the soft layer while measurements along the \hkl[1 1 0]$_\mathrm{pc}$ direction do not show evidence of dragging and are much more square in shape.  
$M_r/M_s$ analysis indicates a value of 0.85 along the \hkl[1 1 0]$_\mathrm{pc}$ and 0.61 along the \hkl[1 0 0]$_\mathrm{pc}$ direction confirming that the easy axis lies along the \hkl[1 1 0]$_\mathrm{pc}$ direction.  
This is in direct contrast with the single layer LSMO film M6 which showed no change between \hkl[1 0 0]$_\mathrm{pc}$ and \hkl[1 1 0]$_\mathrm{pc}$ directions.  
Furthermore the exchange bias for bilayer C12M6 reduces from \SI{.043}{\tesla} to \SI{.005}{\tesla} when the magnetic field is applied along the \hkl[1 1 0]$_\mathrm{pc}$ direction, and a similar reduction is observed for bilayers on LSAT.  

\section{Conclusion}
In conclusion, XA and XMCD spectroscopy reveals the formation of an interfacial \ce{Co^2+} ion-rich layer in LSMO/LSCO bilayers that is magnetically active.  
This interfacial layer only forms for an LSCO thickness of \SI{4}{\nm} and below.  
Multiple other layers form within the LSCO sublayer as a function of increasing film thickness: a MEPS layer forms at the LSCO film - substrate interface with a mixed valence \ce{Co^3+}/\ce{Co^4+} layer above.  
Major hysteresis loops show that the magnetic switching behavior of LSMO/LSCO bilayers significantly changes as a function of LSCO layer thickness, from magnetically decoupled layers for bilayer C12M6 and C8M6 to coupled layers for bilayers C4M6 and C2M6.  
Furthermore, hysteresis loops with the field applied along different crystallographic directions reveal a complex relationship between magnetocrystalline anisotropy, film thickness, and interfaces.  
Minor loop measurements show that for the decoupled bilayers, the exchange spring behavior is anisotropic with very little exchange bias along the \hkl[1 1 0]$_{pc}$ directions.  











