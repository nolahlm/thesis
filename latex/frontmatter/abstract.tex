%!TEX root = ../main.tex


\begin{center}
The Impact of Oxygen Octahedral Rotations on Magnetism in Complex Oxide Bilayers\\
\vspace{2cm}
\textbf{Abstract}
\end{center}
\begin{doublespace}
Complex oxides are a unique class of materials that possess a wide range of functional properties that are highly tunable as a result of the interplay between spin, charge, orbital, and lattice degrees of freedom.  
The complex nature of the interactions between different degrees of freedom leads to novel interfacial behavior and a rich parameter space with interesting functionalities.  
Bilayers of \ce{La_{0.7}Sr_{0.3}MnO3} (LSMO) / \ce{La_{0.7}Sr_{0.3}CoO3} (LSCO) are a prime example of a complex oxide system displaying unique magnetic properties.  
LSMO / LSCO bilayers grown on the cubic substrate \ce{(LaAlO3)_{0.3}(Sr2TaAlO6)_{0.7}} (LSAT) exhibit charge transfer between the LSMO and LSCO layers resulting in the formation of a thin magnetic LSCO layer at the LSMO / LSCO interface.  
The interfacial LSCO layer has a high concentration of \ce{Co^2+} ions and is magnetically coupled with the LSMO layer.  

In this work, the effect of manipulating the lattice degree of freedom on magnetic behavior and charge transfer in LSMO / LSCO bilayers was investigated by growing bilayer films on the orthorhombic substrate \ce{NdGaO3} (NGO), which enhances oxygen octahedral rotations in the film.  
Because magnetic behavior is intimately linked to the bonding geometry, manipulation of oxygen octahedral rotations provides a method to manipulate magnetic behavior in complex oxides.  
The bilayers were grown using pulsed laser deposition followed by structural characterization using x-ray reflectivity and x-ray diffraction to verify structural parameters.  
Diffraction from half-order Bragg peaks was utilized to determine the octahedral rotation pattern, while diffraction simulations were used to obtain quantitative octahedral rotation angles.  
X-ray absorption (XA) was used in order to study the charge transfer phenomena, while x-ray magnetic circular dichroism (XMCD) was employed to determine magnetically active states.  
Finally, vibrating sample magnetometry was used to measure the magnetic switching behavior of the films.  

It was found that the octahedral rotation pattern of LSMO / LSCO bilayers on NGO substrates differs from both the bulk film rotation patterns as well as that of the substrate.  
XA and XMCD results showed that the charge transfer process responsible for \ce{Co^2+} ion-rich layer formation only occurs in bilayers on NGO substrates with LSCO layer thickness of \SI{4}{\nm} and below.  
Furthermore, the lack of any trend in magnetic anisotropy and switching as a function of thickness indicates that a number of competing factors are responsible for the complicated magnetic behavior present in LSMO / LSCO bilayers on NGO substrates.  
\end{doublespace}
