%!TEX root = ../main.tex
\chapter{Conclusion}
\section{Summary}
In this Thesis, the impact of the lattice degree of freedom on interfacial charge transfer and magnetism in \ce{La_{0.7}Sr_{0.3}MnO3} (LSMO) / \ce{La_{0.7}Sr_{0.3}CoO3} (LSCO) bilayers was investigated.  
Both single layers of LSCO or LSMO as well as LSMO / LSCO bilayers were grown on \hkl(1 1 0)$_o$-oriented \ce{NdGaO3} (NGO) substrates which modifies the octahedral rotation pattern in the films.  
Structural characterization via x-ray diffraction (XRD) was used in tandem with x-ray absorption (XA), x-ray magnetic circular dichroism (XMCD), and vibrating sample magnetometry (VSM), to monitor how magnetism is influenced by changing the LSCO sublayer thickness in LSMO / LSCO bilayers.  

It was found that single layer LSCO and the LSCO sublayer in an LSMO / LSCO bilayer take on an octahedral rotation pattern that differs from both the substrate and bulk film pattern.  
For a \SI{15}{\nm} LSCO film, the octahedral rotation angles were found to be \SI{3}{\degree} higher along the \hkl[0 1 0]$_{pc}$ direction compared with the bulk value, \SI{2}{\degree} lower along the \hkl[0 0 1]$_{pc}$ direction, \SI{1}{\degree} lower along the \hkl[0 1 0]$_{pc}$ direction.   

XA and XMCD spectra show that single layer LSCO films on orthorhombic NGO substrates are broken up into a number of electronically distinct layers as a function of film thickness.  
Close to the film-substrate interface is a non-magnetic layer consisting of predominantly \ce{Co^3+} ions.  
This layer has an XA peak shifted to an energy approximately \SI{0.5}{\eV} below the bulk LSCO peak with no XMCD signal.  
Above this non-magnetic layer is a bulk-like layer with mixed \ce{Co^3+} / \ce{Co^4+} ions that is magnetically active and grows with increasing film thickness.  
The LSCO sublayer in LSMO / LSCO bilayers also shows intriguing behavior as a function of layer thickness.  
For \SI{2}{\nm} LSCO layer thickness (sample C2M6), XA and XMCD spectra indicate the presence of the same non-magnetic \ce{Co^3+} ion layer at the film-substrate interface and a magnetically active \ce{Co^2+} ion layer at the LSMO / LSCO interface.   
When the LSCO thickness increases to \SI{4}{\nm} (sample C4M6) an additional \ce{Co^3+} / \ce{Co^4+} ion magnetically active layer forms in between the non-magnetic layer and the \ce{Co^2+} ion-rich layer.  
XA and XMCD spectra for \SI{8}{\nm} (sample C8M6) and \SI{12}{\nm} (sample C12M6) LSCO sublayer thicknesses do not show any \ce{Co^2+} ion signature suggesting that the interfacial \ce{Co^2+} ion-rich layer is not present for these two samples, effectively splitting the series of four samples into two sets, C2M6 and C4M6 which show the \ce{Co^2+} ion-rich layer and C8M6 and C12M6 which do not show the \ce{Co^2+} ion-rich layer.  
These results are in contrast with prior work performed on cubic LSAT substrates where a continuous reduction of \ce{Co^2+} ion signal is observed, remaining present for an \SI{8}{\nm} LSCO sublayer thickness~\cite{Li2016,Kane2019}.  

The complex layer thickness dependence of LSMO / LSCO bilayers results in differences in the magnetic behavior as a function of thickness.  
The LSMO and LSCO sublayers are magnetically coupled in bilayer samples C4M6 and C2M6 leading to a single magnetic switching event for the bilayer, while samples C8M6 and C12M6 show two distinct magnetic transitions because the LSCO and LSMO layers are magnetically distinct.  
Magnetic switching behavior of LSMO / LSCO bilayers on cubic LSAT substrates show the same trend, with the thinner \SI{4}{\nm} and \SI{2}{\nm} LSCO sublayer thickness bilayers exhibiting coupled magnetic behavior and the thicker \SI{8}{\nm} LSCO sublayer showing independent magnetic behavior for LSMO and LSCO layers~\cite{Li2016}.  
Sample C12M6 has a higher $T_c$, as defined by the single $T_c$ in remanent magnetization measurements, than sample C8M6, while sample C4M6 has the highest $T_c$ of all of the bilayers indicating that the magnetic coupling between the LSMO and LSCO layer stabilizes magnetism in LSCO.  
Hysteresis loop measurements along the \hkl[1 0 0]$_{pc}$ and \hkl[0 1 0]$_{pc}$ for the bilayer sample series indicate no distinct in-plane magnetic anisotropy trends as a function of LSCO sublayer thickness indicating that a number of competing interactions are controlling the magnetic behavior in LSMO / LSCO bilayers.  
This observation is in agreement with similar measurements performed on cubic LSAT substrates~\cite{Wynn2015}.  

\section{Future Work}
The absence of a \ce{Co^2+} interfacial layer for LSMO / LSCO bilayers with an LSCO thickness greater than \SI{4}{\nm} may be related to NGO substrate-induced octahedral rotations.  
While substrate induced octahedral rotation patterns tend to permeate throughout LSCO films, the rotation magnitude is not constant: for LSMO / LSCO bilayers grown on LSAT substrates the LSCO octahedral rotation angle in the vicinity of the substrate is suppressed and only reaches its maximum value after approximately 16 unit cells($\sim \SI{6}{\nm}$)~\cite{Byers2019}.  
The case is different for perovskite films on NGO substrates, where rotation magnitude is close to the NGO bulk value of \SI{10}{\degree} close to the substrate and decreases as the film thickness increases~\cite{Yuan2018}.  
This means that for the thinner bilayers C4M6 and C2M6, the octahedral rotation magnitude in LSCO close to the LSMO interface is likely to be larger than in the two thicker bilayers.  
Distorted bond angles, resulting from a larger octahedral rotation magnitude, typically increase resistivity and reduce the strength of exchange interactions~\cite{Moon2014} and it may be expected that a more distorted bond network would therefore minimize interfacial charge transfer, manifesting as the absence of an interfacial \ce{Co^2+} layer for the thinner LSMO / LSCO bilayers.  
Interestingly, the opposite effect is seen where only the thinner bilayers show this interfacial layer and therefore the link between octahedral rotations and interfacial charge transfer is more complex than a change in octahedral rotation magnitude.  

In order to further elucidate the mechanism responsible for the absence of charge transfer in the thicker bilayer films, a detailed thickness dependent study of octahedral rotation patterns and magnitude is required.  
Analysis of half-order Bragg peaks may not be possible for the thinner bilayers because the low film volume will reduce the already low diffraction intensity; therefore, other techniques to analyze the octahedral rotations in these films are required.  
TEM studies can be performed to directly observe the change of atomic positions across the multiple interfaces in the heterostructures, although the creation of a suitable sample for TEM studies may alter the structural integrity of the bilayer.  
A promising non-destructive technique, coherent Bragg rod analysis (COBRA), provides a reconstruction of the 3D electron density map across interfaces and could also be applied as a method to track changes in octahedral rotation magnitude and pattern~\cite{Kumah2008}.  
Either of these techniques may reveal distinct differences in octahedral rotation patterns and magnitude as a function of film thickness that explain the observed charge transfer trend.  

When performing these thickness dependent studies, it would be crucial grow and investigate additional samples with LSCO thicknesses between \SI{4}{\nm} and \SI{8}{\nm} to assist in the identification of a critical LSCO thickness for the  formation of a \ce{Co^2+} ion-rich layer.  
Furthermore, LSMO / LSCO bilayers on LSAT substrates with a 1:1 thickness ratio tend to break the trend of a reduction in \ce{Co^2+}-ion signal as thickness is increased and it would therefore be interesting to see if growth on NGO substrates follows the same trend.  

Finally, resonant soft x-ray reflectivity (RSXRR) can be employed in order to obtain depth-resolved electronic and magnetic information from LSMO / LSCO bilayers.  
RSXRR takes advantage of reflectivity curves measured at several resonant energies where scattering from specific ions is enhanced~\cite{Fink2013}.    
RSXRR can be employed on any of the single layer or bilayer samples analyzed in this work and can provide a measure for the length scale of both charge transfer and MEPS layer formation in LSMO / LSCO bilayers.   